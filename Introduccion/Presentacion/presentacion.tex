\section{Presentación}
% Como se puede ver en \cite{Baz} ... %%% EJEMPLO CITAS
La Organización Mundial de la Salud ha destacado que la depresión constituye un problema importante de salud pública, ya que más del 4\% de la población mundial vive con depresión y los más propensos a padecerla son las mujeres, los jóvenes y los ancianos.\\\\
A nivel mundial, este trastorno se ha convertido en la cuarta causa de discapacidad en cuanto a la pérdida de años de vida saludables. En México, la depresión ocupa el primer lugar de discapacidad para las mujeres y el noveno para los hombres. Además, se estima que 9.2\% de la población ha sufrido depresión, que una de cada cinco personas sufrirá depresión antes de los 75 años y que los jóvenes presentan tasas mayores.\\\\
La depresión se caracteriza por la presencia de tristeza, pérdida de interés o placer, sentimientos de culpa o falta de autoestima, trastornos del sueño o del apetito, sensación de cansancio y falta de concentración. Este trastorno puede llegar a hacerse crónico o recurrente y en su forma más grave, puede conducir al suicidio.\\\\
El INEGI documentó que 34.85 millones de personas se han sentido deprimidas, de las cuales 14.48 millones eran hombres y 20.37 millones eran mujeres. Un aspecto muy a tener en cuenta es que del total de personas que se han sentido deprimidas, únicamente 1.63 millones toman antidepresivos. \cite{descifremos}\\\\ 
% https://www.gob.mx/cms/uploads/attachment/file/280081/descifremos15-2.pdf
